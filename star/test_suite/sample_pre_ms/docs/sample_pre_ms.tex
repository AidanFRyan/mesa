\documentclass{article}
%\usepackage{geometry}
% \geometry{top = 1in, bottom = 1in, left = 1in, right = 1in}
\usepackage[top = 0.7in, bottom = 0.7in, left = 0.7in, right = 0.7in]{geometry}
\usepackage{amsmath,amssymb,amsthm,mathrsfs}
\usepackage{graphicx}
\usepackage{bm}
\usepackage{float}
\usepackage[font=footnotesize,labelfont=bf]{caption}

\usepackage{fancyhdr}
\pagestyle{fancy}
\rhead{\footnotesize {08/23/2012 ; MESA version 4299} }
\chead{\footnotesize {Authors: Jared Brooks, Lars Bildsten, Bill Paxton} }
\lhead{\footnotesize {mesa/star/test\_suite/sample\_pre\_ms} }

\begin{document}
	
	\begin{center}
		\begin{Large}
		       \textbf{SAMPLE PRE MS}\\
		\end{Large}
	\end{center}

        This test case is to make sure that new versions of \texttt{MESA} can still create and run pre-main sequence models.  The file \texttt{src/run\_star\_extras.f} contains a list of three masses (\texttt{Ms = (/ 20d0, 0.2d0, 2d0 /)} and three metalicities (\texttt{Zs = (/ 0.04d0, 0.0d0, 0.02d0 /)}), and \texttt{MESA} creates pre-main sequence models for each of the nine permutations of mass and metallicity and will run each one for 100 steps.  The inlist for this test case uses only default values, except for \texttt{kappa\_file\_prefix = 'gs98'} and \texttt{max\_number\_backups = 12 ; max\_number\_retries = 12}.  If each runs without any problems, the terminal output at the end should read \texttt{``finished sample pre-ms''}.\\

\end{document}
